 
\documentclass[12pt, letterpaper, twoside]{article}
\usepackage[utf8]{inputenc}
\usepackage{hyperref}
\hypersetup{
    colorlinks=true,
    linkcolor=blue,
    filecolor=magenta,      
    urlcolor=cyan,
}
 
\urlstyle{same}
\title{Pygame Mini-Project Documentation}
\author{Pragati Kumar Singh}
\date{June 2017}
\begin{document}
\maketitle
\tableofcontents
\section{Introduction}
 First of all, I have taken the project-named Pygame-Miniproject under guidance of 
Milind Luthra from Programming Club, IIT Kanpur. This project mainly focuses on the 
Python and its  open source Pygame Library.
\section{Preparation}

This section tells about how I started to head into the project and also what tasks 
I completed before heading into the project.
\subsection{Getting in touch with Python}
I started learning python from the book 
\emph{\href{http://learnpythonthehardway.org/book/}{Learn Python The Hard Way}}.

The book is titled "Hard Way" which means by "practicing hard". The book is divided 
in small exercises and, one has to type in and run the code, and debug if error 
occured.
I have typed almost all the exercises needed for the project (upto around 45) and 
made each code run perfectly. All the codes typed by me can be found 
\href{http://github.com/pragati4721/LPTHW}{here}.

Most of the keywords and syntax are easy enough to understand. I took me some time 
to understand some of the syntax such as self,init method,super.

\subsection{Getting in touch with Pygame Library}
Python comes pre-installed in Linux machines.

\noindent For installing pip which is useful for downloading many of the free open 
source libraries of Python, type the follwing command in terminal and give the 
password when prompted.

sudo apt install pip

\noindent Then for installing pygame library use the command:

sudo pip install pygame

\noindent Then I watched Pygame Tutorial from Youtube. Whose link can be found 
\href{https://www.youtube.com/playlist?list=PL6gx4Cwl9DGAjkwJocj7vlc_mFU-4wXJq}{here
}. Along with that tutorial , I created my first game with the help of pygame 
library in Python, whose code and files needed to run the game can be found 
\href{http://github.com/pragati4721/LearningPygame/}{here}. 
\subsection{Learning to use Git and Github}
Git is something which is used to track the modifications of files or 
repositories(or just say folders).
\noindent\href{http://github.com}{Github} is online platform for the same. And also 
both can be synced very easily. To clone a github repository on a machine with git 
installed just type the command :

git clone link-for-cloning-online-repo

\noindent Link for cloning a online repo is present in the online repo itself. If 
changes are made in online repo, then for syncing again, we have to open terminal in 
the cloned directory an type in the command:

git pull

\noindent After modifying/adding files in clone repo, to make changes online ,we 
have to type in 3 commands:

git add .

git commit -m "any comment you want"

git push

\noindent When we type git push user name and password for the online github account 
is required.


\subsection{Using Latex}
\LaTeX{} is a tool used to create professional-looking documents. I needed to learn 
this because I had to keep the record of what I done during and before the project, 
so that it may help someone in future, who is taking a similar project. One can 
learn latex from \href{http://www.sharelatex.com/learn/}{here}. One can create his 
\LaTeX{} project online after  creating an online account on 
\href{http://www.sharelatex.com/}{ShareLatex.com} or some GUI softwares such as 
\emph{Texmaker} (for Linux) or any other available. Even this Document is created 
through \LaTeX{} and its code can be found 
\href{http://github.com/pragati4721/pygame-miniproject/Documentation}{here}.

\section{Conclusion}
\end{document}
